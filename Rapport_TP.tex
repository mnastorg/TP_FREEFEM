\documentclass[11pt,a4paper]{report}

\input{/home/matthieu/Documents/LATEX/preambule_master.tex}

\begin{document}
	\pagedegarde{Rapport TP Freefem ++}{Nastorg Matthieu \& Philibert Thomas}
	
	\section*{Question 1 :}
		
		On s'intéresse à l'étude par éléments finis, via le langage de programmation \bsc{FREEFEM++}, du problème suivant : 
			
		\[
			\left\{
			\begin{array}{ll}
			- \Delta u = f & \mbox{ si } u \in \Omega \\
			u = 0 & \mbox{ si } u \in \partial\Omega
			\end{array}
			\right.
		\]
		
		Ici, nous avons choisi la fonction : $x*(1-x)*y*(1-y)$. La résolution de ce problème nous est donné dans les fichiers fournis.
		
		On nous demande, d'évaluer l'erreur sur la semi-norme $H^1$ définie par : 
		\[
			E_{H^1} = \displaystyle \left(\int_{\Omega} |\nabla u - \nabla u_h|^2 \dx \right)^{\frac{1}{2}}
		\]
		
		On affiche, ci-après, le graphique d'erreur $L^2$ et $H^1$ relatif à la fonction. L'erreur est affichée en fonction du maillage utilisé, qui est de plus en plus fin.
		
		\begin{figure}[H]
			\centering
			\includegraphics[scale=0.8]{output1/laplace_errors_umfpack.png}
		\end{figure}
		
		Nous allons, ci-après, donner le tableau des minimum et maximum des solutions approchées et exactes en fonction du pas du maillage.
		
		
		\begin{center}
			\begin{tabular}{|c|c|c|c|c|}
				\hline
				pas & minimum approché & minimum exact & maximum approché & maximum exact \\
				\hline 
				$0.1$ & $8.6*10^{-63}$ & $0$ & $0.0622894$ & $0.0625$\\ 
				\hline 
				$0.05$ & $2.4*10^{-63}$ & $0$ & $0.0624948$ & $0.0625$\\ 
				\hline 
				$0.025$ & $6.4*10^{-64}$ & $0$ & $0.0624861$ & $0.0625$  \\ 
				\hline 
				$0.0125$ & $1.6*10^{-64}$ & $0$ & $0.0624961$ & $0.0625$\\ 
				\hline 
				$0.00625$ & $4.3*10^{-65}$ & $0$ & $0.0624987$ & $0.0625$ \\ 
				\hline 
				$0.003125$ & $1.05*10^{-65}$ & $0$ & $0.0624996$ & $0.0625$ \\ 
				\hline 
			\end{tabular} 
		\end{center}
			
		Nous pouvons constater que les valeurs min et max se rapprochent de la solution exacte plus le maillage s'affine.
		
		\section*{Question 2}
		
			\begin{figure}[H]
			\centering
			\includegraphics[scale=0.8]{output2/laplace_errors_umfpack.png}
		\end{figure}
	
	
	\begin{center}
		\begin{tabular}{|c|c|c|c|c|}
			\hline
			pas & minimum approché & minimum exact & maximum approché & maximum exact \\
			\hline 
			$0.1$ & $1*10^{-31}$ & $0$ & $1.9$ & $2$\\ 
			\hline 
			$0.05$ & $5*10^{-32}$ & $0$ & $1.95$ & $2$\\ 
			\hline 
			$0.025$ & $2.5*10^{-32}$ & $0$ & $1.975$ & $2$  \\ 
			\hline 
			$0.0125$ & $1.25*10^{-32}$ & $0$ & $1.9875$ & $2$\\ 
			\hline 
			$0.00625$ & $6.25*10^{-33}$ & $0$ & $1.99375$ & $2$ \\ 
			\hline 
			$0.003125$ & $3.125*10^{-33}$ & $0$ & $1.99687$ & $2$ \\ 
			\hline 
		\end{tabular} 
	\end{center}
	

		\section*{Question 3}
		
			\begin{figure}[H]
			\centering
			\includegraphics[scale=0.8]{output3/laplace_errors_umfpack.png}
			\end{figure}
		
		
		\begin{center}
			\begin{tabular}{|c|c|c|c|c|}
				\hline
				pas & minimum approché & minimum exact & maximum approché & maximum exact \\
				\hline 
				$0.1$ & $-0.248208$ & $-0.25$ & $0.250697$ & $0.25$\\ 
				\hline 
				$0.05$ & $-0.250084$ & $-0.25$ & $0.250316$ & $0.25$\\ 
				\hline 
				$0.025$ & $-0.249807$ & $-0.25$ & $0.250188$ & $0.25$  \\ 
				\hline 
				$0.0125$ & $-0.249946$ & $-0.25$ & $0.250057 $ & $0.25$\\ 
				\hline 
				$0.00625$ & $-0.249983$ & $-0.25$ & $0.250023$ & $0.25$ \\ 
				\hline 
				$0.003125$ & $-0.249995$ & $-0.25$ & $0.250006$ & $0.25$ \\ 
				\hline 
			\end{tabular} 
		\end{center}

		\section*{Question 4}
		
			\begin{figure}[H]
				\centering
				\includegraphics[scale=0.8]{output4/laplace_errors_umfpack.png}
			\end{figure}
		\section*{Quesiton 5}
		
			\begin{figure}[H]
				\centering
				\includegraphics[scale=0.8]{output5/laplace_errors_umfpack.png}
			\end{figure}
		
		\section*{Question 6}

\end{document}